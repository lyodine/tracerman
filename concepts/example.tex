\documentclass{article}
\usepackage{hyperref}
\usepackage{color}

\title{Test}
\author{Authorman}
\date{today}
\begin{document}

% 







\ExplSyntaxOn
%define \my_concat:nn

\newcommand{\tracerman_error}[1]{\textcolor{red}{[TRACERMAN~ERROR:~#1]}}

\makeatletter
    \newcommand\is_label_defined[1]{
        \@ifundefined{r@#1}{no}{yes}
    }
\makeatother

% \bool_if:NTF \l_tmpa_bool {WHAT} {
%       WHAT
% }


%use \my_concat:nn



% reference
% \tracerman_write_label:n {a} %result: a, b
% \tracerman_ref_label:n {a} %result: a, b



% \tracerman_label_if{what}{true}{false}

% the command attempts to 
% determine the type of the new function, which shoudl reference    bool_od`

% [function] \tracerman_label_if:nNN (label, ifcode, elsecode)
% @param label the name of a label
% @param ifcode code that executes if label is defined
% @param elsecode code that executes if label is not defined
\bool_new:N \l_tracerman_is_label_defined_bool
\str_new:N \l_tracerman_result_of_checking_if_label_is_defined
\cs_set:Npn \tracerman_label_if:nNN #1#2#3 {
    % I use an extremely convoluted hack
    %   to implement conditional execution, depending on whether a label
    %   is already defined.
    % Step 1: define local variables
    % Step 2: generate variant \str_set:Nx. This variant expands its second
    %   argument recursively. This behaviour  is necessary, because
    %   said "second argument" checks if a label is defined.
    \cs_generate_variant:Nn \str_set:Nn {Nx}
    % Step 3: Assign the result, which can be either 'yes' or 'no', to a variable.
    \str_set:Nx \l_tracerman_result_of_checking_if_label_is_defined {\is_label_defined{#1}}
    % Step 4: Define a new variable whose value is 'yes'. This is necessary because
    %   I believe there must be a better way to do it. As it stands, this is
    %   what 10 hours of attempts get me.
    \str_set:Nn \yes {yes}
    % Step 5: Compare the value in \l_tracerman_result_of_checking_if_label_is_defined
    %   against the value in 'yes'. Set the variable
    %   \l_tracerman_is_label_defined_bool accordingly to the result.
    \str_if_eq:NNTF \yes \l_tracerman_result_of_checking_if_label_is_defined {
        \bool_set_true:N \l_tracerman_is_label_defined_bool
    }{
        \bool_set_false:N \l_tracerman_is_label_defined_bool
    }

    % Step 6: Finally, the boolean variable \l_tracerman_is_label_defined_bool
    %   makes conditional execution poossible.
    \bool_if:NTF {\l_tracerman_is_label_defined_bool} {#2} {#3}
}

\cs_set:Npn \tracerman_write_label:n #1 {
    #1\label{#1}
}

\cs_set:Npn \tracerman_ref_label:n #1 {
    \tracerman_label_if:nNN {#1} {
        \hyperref[#1]{#1}
    } {
        \tracerman_error{Label~`#1'~is~referenced~but~not~defined.}
    }
}

\tracerman_write_label:n {mylabel}\\

\tracerman_ref_label:n {mylabel}\\
\tracerman_ref_label:n {labelthatdoesnotexist}\\


\ExplSyntaxOff




% \ExplSyntaxOn



% \ExplSyntaxOff




% define \tracerman_write_label
% \cs_set:Npn \tracerman_write_label:n #1 {
%     #1
% }
% Functions:

% register_item(name)
% if (is_name_defined(name)) then
%     raise warning
% else
%     add name to name_set
%     write_label(name)
% end if

% ref_item(name)
% if (not is_name_defined(name)) then
%     raise warning("referenced name not defined")
% else
%     ref_label(name)
% end if

% link_item(name1)(name2)
% if (not (is_name_defined(name1)
%         or is_name_defined(name2))) then
%     raise warning("referenced names not defined")
% else
%     add to list (get_target_list_of(name1))
% end if

% get_target_list_of(name1)
%...



\end{document}